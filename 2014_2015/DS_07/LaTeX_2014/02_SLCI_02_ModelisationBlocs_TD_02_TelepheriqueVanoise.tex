\documentclass[10pt]{article}
\input{style/coursHeadings}
\input{style/programHeadings}
\input{style/macros_SII}
\input{style/macros_Titres}
\input{style/macros_Frames}

%Si le boolen xp est vrai : compilation pour xabi
%Sinon compilation Damien
\newboolean{xp}
\setboolean{xp}{true}

\newboolean{prof}
\setboolean{prof}{true}

\newif\ifprof
%\proftrue
\proffalse

\newboolean{td}
\setboolean{td}{true}

\usepackage[%
    pdftitle={},
    pdfauthor={Xavier Pessoles},
    colorlinks=true,
    linkcolor=blue,
    citecolor=magenta]{hyperref}

\def\discipline{Sciences Industrielles de l'Ingénieur}

\def\xxtitre{\ifthenelse{\boolean{xp}}{CI 2 -- SLCI : Systèmes Linéaires Continus et Invariants -- Analyser, Modéliser, Résoudre}{}}

\def\xxsoustitre{\ifthenelse{\boolean{xp}}{
Chapitre 3 -- Modélisation des SLCI par Schémas Blocs}{
}}


\def\xxauteur{\ifthenelse{\boolean{xp}}{
\noindent Xavier \textsc{Pessoles}}{
}}


\def\xxpied{\ifthenelse{\boolean{xp}}{
CI 2 : SLCI \\
Ch 3 : Modélisation par schémas blocs -- TD -- \ifthenelse{\boolean{prof}}{P}{E}%
}{
}}

\def\xxcathegorie{\ifthenelse{\boolean{xp}}{
2013 -- 2014 \\
Xavier \textsc{Pessoles}}{
Informatique - Cours}}





%---------------------------------------------------------------------------


\begin{document}

\ifthenelse{\boolean{xp}}{\input{style/enteteXP}}{\input{style/enteteDI}}



%\renewcommand{\baselinestretch}{1.2}
%\setlength{\parskip}{2ex plus 0.5ex minus 0.2ex}



\begin{comp}
\noindent \textbf{Modéliser :} un système étant fourni, et les exigences définies, l’étudiant doit être capable de proposer un modèle de connaissance du système ou partie du système à partir des lois physiques.

\noindent \textit{Mod2 -- C2.2} Représentation par fonction de transfert (formalisme de Laplace).

\noindent \textbf{Résoudre :} à partir des modèles retenus, mettre en œuvre une méthode de résolution.
 
\noindent \textit{Rés -- C5.1 :} Simplification d’un schéma bloc : déplacement d’un sommateur, déplacement d’un point de prélèvement.
\end{comp}

\section*{Téléphérique Vanoise Express}

\begin{flushright}
\textit{D'après concours E3A -- PSI -- 2014.}
\end{flushright}

\ifprof
\else

\begin{minipage}[c]{.8\linewidth}
Depuis 2003, le téléphérique Vanoise Express relie les domaines de La Plagne et des Arcs séparés, par les airs, d'une distance de 1 830 mètres.
\begin{obj} 
Le but de ces travaux est de vérifier que, conformément au cahier des charges, l'écart de traînage, en vitesse, en l'absence de perturbation est nul. 
\end{obj}

 On donne un extrait du cahier des charges.
 
\end{minipage} \hfill
\begin{minipage}[c]{.15\linewidth}
\begin{center}
\includegraphics[width=\textwidth]{images/ve}
\end{center}
\end{minipage}

\begin{minipage}[c]{.15\linewidth}
\begin{center}
\includegraphics[width=\textwidth]{images/uc}

\textit{Diagramme des cas d'utilisation}
\end{center}
\end{minipage} \hfill
\begin{minipage}[c]{.15\linewidth}
\begin{center}
\includegraphics[width=\textwidth]{images/contexte}

\textit{Diagramme de contexte}
\end{center}
\end{minipage} \hfill
\begin{minipage}[c]{.6\linewidth}
\begin{center}
\includegraphics[width=\textwidth]{images/Exigences}

\textit{Diagramme partiel des exigences}
\end{center}
\end{minipage}

\vspace{.25cm}

On donne deux schémas de principe illustrant le fonctionnement d'une ligne de téléphérique :

\vspace{.25cm}

\begin{minipage}[c]{.49\linewidth}
\begin{center}
\includegraphics[width=\textwidth]{images/Schema_01}

\textit{Schéma de principe d'une ligne de téléphérique}
\end{center}
\end{minipage} \hfill
\begin{minipage}[c]{.49\linewidth}
\begin{center}
\includegraphics[width=\textwidth]{images/Schema_02}

\textit{Schéma de la salle des machines}
\end{center}
\end{minipage}
\vspace{.25cm}




\subsection*{Modélisation du moteur électrique à courant continu}
On notera :
\begin{itemize}
\item $F(p)$ la transformée de Laplace d’une fonction du temps $f(t)$;
\item $u(t)$ : la tension d’alimentation des moteurs ;
\item $i(t)$ : l’intensité traversant un moteur ;
\item $e(t)$ : la force contre électromotrice d’un moteur ;
\item $\omega_m (t)$ : vitesse de rotation d’un moteur ;
\item $c_m (t)$ : couple d’un seul moteur ;
\item $c_r (t)$ : couple de perturbation engendré par le poids du téléphérique dans une pente et par l’action du vent, ramené sur l’axe des moteurs.	 
\end{itemize}


Hypothèses : on suppose les conditions initiales nulles et que les deux moteurs fonctionnent de manière parfaitement identique. 


On note : 
\begin{minipage}[c]{.49\linewidth}
\begin{itemize}
\item $L=0,59 \; \text{mH}$, inductance d’un moteur ;
\item $R=0,0386 \; \Omega$ : résistance interne d'un moteur ;
\item $f=6 \; \text{N}\cdot \text{m}\cdot \text{s/rad}$ : coefficient de frottement visqueux équivalent ramené sur l’axe des moteurs;
\item $J=800 \; \text{kg}\cdot \text{m}^2$ : moment d’inertie total des pièces en rotation, ramené sur l’axe des moteurs.
\end{itemize}
\end{minipage}\hfill
\begin{minipage}[c]{.49\linewidth}
\begin{center}
\includegraphics[width=.8\textwidth]{images/SchemaElectrique}

\textit{Schéma électrique du moteur à courant continu}

\end{center}
\end{minipage}

\vspace{.25cm}
Le fonctionnement électromécanique des deux moteurs est décrit par les équations suivantes :
\begin{itemize}
\item $c_m (t)=k_T \cdot i(t)$ avec $k_T=5,67 \;\text{Nm/A}$ (constante de couple d’un moteur) ;
\item $e(t)=k_E \cdot \omega_m (t)$ avec $k_E=5,77 \; \text{Vs/rad}$ (constante électrique d’un moteur).
\end{itemize}

L’application des théorèmes de la dynamique permet d’écrire que : 
$$ 2\cdot c_m (t)-c_r (t)=J\omega_m (t)+f\cdot \omega_m (t)$$

L’application de la loi des mailles dans le schéma électrique se traduit par l’équation suivante :
$$u(t)=e(t)+Ri(t)+L \cdot \dfrac{di(t)}{dt}$$


\subparagraph{}
\textit{Exprimer chacune des 4 équations dans le domaine de Laplace.}
\ifprof
\begin{corrige}
\end{corrige}
\else
\fi

\begin{center}
\includegraphics[width=.9\textwidth]{images/SchemaBloc_01}
\end{center}

\subparagraph{}
\textit{Exprimer les fonctions de transfert $G_1(p)$, $G_2(p)$, $G_3(p)$ et $G_4(p)$.}
\ifprof
\begin{corrige}
\end{corrige}
\else
\fi


\subsubsection*{On considère que $C_r (p)=0$ et $U(p)\neq 0$.}

\subparagraph{}\textit{Calculer la fonction de transfert $F_1 (p)=\dfrac{\Omega_m (p)}{U(p)}$ en fonction de $G_1(p)$, $G_2(p)$, $G_3(p)$, $G_4 (p)$.}
\ifprof
\begin{corrige}
\end{corrige}
\else
\fi

\subsubsection*{On considère que $C_r (p)\neq0$ et $U(p)=0$.}
\subparagraph{}
\textit{Retracer le schéma bloc.}
\ifprof
\begin{corrige}
\end{corrige}
\else
\fi

\subparagraph{}
\textit{Calculer la fonction de transfert $F_2 (p)=\dfrac{\Omega_m (p)}{C_r (p)}$ en fonction de $G_1(p)$, $G_2(p)$, $G_3(p)$, $G_4 (p)$.}

\ifprof
\begin{corrige}
\end{corrige}
\else
\fi

\subsubsection*{On considère que $C_r (p)\neq0$ et $U(p)\neq0$.}

\subparagraph{}
\textit{Exprimer $\Omega_m(p)$ en fonction de $F_1 (p)$, $F_2 (p)$, $C_r (p)$ et $U(p)$.}
\ifprof
\begin{corrige}
\end{corrige}
\else
\fi

\subsection*{Modélisation des deux moteurs à courant continu}

\vspace{.25cm}

\begin{minipage}[c]{.4\linewidth}
Dans certaines conditions particulières, il est possible de mettre le moteur sous la forme ci-dessous avec :
\begin{itemize}
\item $B=297,4\; \text{N}\cdot \text{m⁄V}$;
\item $D=5,8\;10^{-4} \;  \text{rad⁄s}\cdot \text{N}\cdot\text{m}$;
\item $T=0,47\;\text{s}$.
\end{itemize}
\end{minipage}\hfill
\begin{minipage}[c]{.59\linewidth}
\begin{center}
\includegraphics[width=.9\textwidth]{images/SchemaBloc_02}
\end{center}
\end{minipage}

\vspace{.25cm}

La motorisation modélisée ci-dessus est insérée dans une boucle d’asservissement.

\begin{center}
\includegraphics[width=\textwidth]{images/SchemaBloc_03}
\end{center}


La consigne de vitesse $v_c(t)$ est donnée en entrée. Elle est convertie en une tension $\rho_c(t)$ avec le gain $F$. Une génératrice tachymétrique de gain $\mu =0,716 \; \text{V}\cdot \text{s/rad}$ transforme la vitesse de rotation $\omega_m(t)$ du moteur en une tension $\rho_m(t)$. Un correcteur de fonction de transfert $C(p)$ corrige la différence $\varepsilon (t)= \rho_c(t) -  \rho_m(t)$ et l’envoie à un amplificateur de gain $A$, qui alimente les deux moteurs électriques. La vitesse de rotation des moteurs $\omega_m(t)$ est transformée en vitesse du téléphérique $v(t)$ avec le gain $E$.

\subparagraph{}
\textit{Déterminer l’expression du gain $E$. Faire une application numérique.}
\ifprof
\begin{corrige}
\end{corrige}
\else
\fi



\subparagraph{}
\textit{Déterminer l’expression du gain $F$ pour que $\varepsilon (t)=0$ entraîne $v_c(t)=v(t)$. Faire une application numérique.}
\ifprof
\begin{corrige}
\end{corrige}
\else
\fi


On considère que $C_r (p)=0$ et $A\cdot C(p)=1$. On réalise les applications numériques. Le schéma bloc peut alors se mettre sous la forme : 
$$
H(p)=\dfrac{V(p)}{V_c (p)}= \dfrac{\dfrac{AK_1}{1+K_1A}}{1+\dfrac{T}{1+K_1A} p}
$$

Avec 
$K_1=0,123 \;\text{SI}$ et $T=0,47\;\text{s}$.

On sollicite le système avec en entrée un échelon unitaire. C’est-à-dire que $v_c(t)$ est un échelon unitaire. 


\subparagraph{}
\textit{Calculer $V(p)$ puis $V_c (p)$. }
\ifprof
\begin{corrige}
\end{corrige}
\else
\fi



\subparagraph{}
\textit{Déterminer la valeur finale de $v(t)$ en fonction de $A$. }
\ifprof
\begin{corrige}
\end{corrige}
\else
\fi



\subparagraph{}
\textit{Si tel n’est pas le cas, calculer $A$ pour que la valeur finale de $v(t)$ soit 1.}
\ifprof
\begin{corrige}
\end{corrige}
\else
\fi

On sollicite le système avec une entrée en rampe de pente 1. 

\subparagraph{}
\textit{Montrer que $V(p)$ peut se mettre sous la forme $\dfrac{H(p)}{p^2}$. }
\ifprof
\begin{corrige}
\end{corrige}
\else
\fi



\subparagraph{}
\textit{Déterminer la valeur finale, la valeur initiale et la tangente à l’origine de $v(t)$ en fonction de $A$.}
\ifprof
\begin{corrige}
\end{corrige}
\else
\fi


\subparagraph{}
\textit{Mettre $V(p)$ sous la forme suivante : }
$$
V(p)=\dfrac{A}{p}+\dfrac{B}{p^2} +\dfrac{C}{...}
$$

\ifprof
\begin{corrige}
\end{corrige}
\else
\fi



\subparagraph{}
\textit{Déterminer $v(t)$.}
\ifprof
\begin{corrige}
\end{corrige}
\else
\fi





\end{document}


